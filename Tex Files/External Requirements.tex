\documentclass[11pt]{article}
\usepackage{parskip}

\begin{document}
\section{External Requirements}

\subsection{Navigation Module}
\subsubsection{Hardware Interface Requirements}
  
The navigation module itself will be receiving a lot of data from the GIS module and won’t specifically have many hardware requirements,  aside from running on a device, or through a platform that has sufficient processing power to execute the algorithms that calculate and track specific routes as well as requiring a screen to output an interface to the user. The GIS module on the other hand will require that the device has WiFi and GPS chips for location triangulation.
  
 \subsubsection{Software Interface Requirements}
  
Depending on the technologies used to aid in the implementation of the system, the navigation module may use API plugins to help realise its functionality. There are many such APIs available that allow for custom GIS data and can provide turn by turn navigation, calculation of shortest routes and can run on many popular systems such as android, iOS and web-based platforms. The navigation module could interface with one or many of these APIs with additional code perhaps being required to streamline the integration with the NavUP system. Support for accessibility software such as screen readers for visually disabled user and other accessibility software will also be implemented.
  
 \subsubsection{User Interface Requirements}
  
The user interface for the navigation module should provide an easy means for the user to select a destination they would like to travel to and be presented with a number of possible routes. This could be accomplished by immediately presenting the user with a map of campus localised to their current position. A search bar along the top will allow the user to search for a destination from which they’ll be shown a zoomed out view of campus with the possible routes highlighted on the map. They would then be able to select a route and would be presented with screen or voice based, turn by turn instructions to reach their desired destination. An additional menu button could bring up options to set the user’s route preference or access their saved routes. The interface can also provide text-to-speech functionality for users with disabilities.
  
 \subsubsection{Communication Interface Requirements}
  
The Navigation module’s communication requirements should be described by the other modules with which it interacts. The navigation module will create a GIS request that will be sent to the GIS module which will in turn use the required communication protocols to access and retrieve map data from the GIS database. A GIS request could also be for utilising the device’s GPS chip and communicating with a satellite to determine device position. Alternatively the WiFi chip could be used and based on its communication, via network protocols, with nodes in the UP network, the device position could theoretically be determined.

\subsection{Data Module}

\subsubsection{Hardware Interface Requirements}

The Data Module requires the NavUP system to have a server and a database(s). The database(s) must be able to send and retrieve streams form users. The server must be able to communicate with the database(s) in order to allow the users to have an interaction with the database(s).
Both the server and the database(s) need to be equipped with powerful and up to standard hardware that allows for fast processing of mass amounts of streams of data.

\subsubsection{Software Interface Requirements}

The data module requires the NavUP system to have the appropriate software residing on both the server and the database(s). The software must be able to control the hardware and make sure all operations happen in the fastest and most optimal way.The software must monitor all the operations in order to derive trends and improve the way that it handles requests form users as well as the way it interacts with streams of data.

\subsubsection{Event Requirements}

The user is required to poses a smart which has the capability to connect to a ISP or preferably the University of Pretoria’s Wifi for a stronger connection. The device the user to be on campus for maximum effectivity.

\end{document}
