\documentclass[11pt]{article}
\usepackage{parskip}

\begin{document}
\section{Design Constraints}

\subsection{Navigation Module}
Design constraints for the Navigation module include:
 \begin{itemize}

      \item The system using the Navigation module must have WiFi support with an intenet connection to be able to get real time updates of the map and route changes.
      \item The system should also include suppot for a GIS to be able to find the user's current location at all times.
       \item The system should also include a monitor to display the mapp and route information to the user and the system will also require a processor strong enough to calculate the different routes to the user's selected destination.
        \item The Navigation module should also be able to record the route the user is taking or a route the userr has chosen and send that information to the data module to store the route as a preference.
      
  \end{itemize}
  
\subsection{Data Module}
Some design constraints for the data module include:
  \begin{itemize}

      \item The correct data types for information to be entered into the system database need to be carefully chosen. 
    Data types need to fit accordingly as to not watse overall system storage and downgrade the systems performance.

      \item The system server need to be equipped with a quality processor in order to process mass amounts of upstream and downstram data at the same time.
    Failure to implement a quality processor, might lead to the system crashing under the pressure of mass amounts of requests from users.

    \item The server hardware need be equipped with the correct and necessary security measures.
    Failure to equip the server with the correct and necessary security measures might lead to the system database being at risk,
    which can lead to sensitive user information being at risk for theft or even loss.

  \end{itemize}
  
  \subsection{Users Module}
  
  \subsection{Points of Interest Module}
  
  \subsection{Fitness Module}

  \subsection{Events Module}
  *Events can be stored on the devices calendar so without the presence of an ISP or internet source
  of any kind the app will show very little effectivity with events that require dynamic addition during
  the year.
  *Event calendar must be able to link to the users email account in order to send them updates on events
  that may appear in order to keep them connected in a scenario where the user wants to view events happening 
  without having to install the app on a device that isn’t theirs.
  *Linkage of a client’s history to future event selection, for this to be possible the program must be 
  recorded this being their likes, types of events attended, and subscription to an event category and 
  then receiving notifications regarding them.

  
  
 \end{document}
