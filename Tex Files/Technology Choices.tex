\documentclass[11pt]{article}
\usepackage{parskip}

\begin{document}
\section{Technology Choices}

\subsection{Navigation Module}
Leaflet is an open-source JavaScript library that provides mapping functions for mobile developers. It runs efficiently on all major mobile and desktop platforms and therefore satisfies our requirement of the NavUP system needing to have Android, iOS and a web-based frontend. Leaflet can work with custom GIS data so will allow for the use of our provided GIS mapping of the University of Pretoria’s campus. The API at a basic level is mainly for displaying map data however there are a number of plugins available to expand the functionality. One of these is called Leaflet Routing Machine and provides a graphical extension to leaflet, displaying routes between a start and end point. Leaflet routing machine provides additional functionality for backend routing engines, one of these being OSRM (Open Source Routing Machine). OSRM uses extremely fast methods for shortest path routing and provides turn by turn navigation instructions to the user. These three technologies already interface with one another, are all open source and satisfy most of the constraints of the NavUP Navigation module. Additional implementation could ensure that the technologies integrate well into the NavUP system.

\subsection{User Module}
For this module to work effectivley, we must me able to query and add users details to the database. We will use SQL for databases for all the platforms. PHP will help in retrieving data (Mostly for Web). The SQLiteOpenHelper import (IOS/Android) will help us use SQL query and add fuctions for both platforms.
For the user Interface and nevigation through pages we will use Android studio and Xcode for IOS or we could use a cross platform tool for all the platforms(e.g PhoneGap). 

\subsection{Data Module}
Node.js is an asynchronous event driven javascript runtime non-blocking Input Out put model. It is used to create a scalable and effiecent server for network applications. Using Node.js as a I/O model server will create an efficient server which handles realtime interactions much faster than other server languages. It would be beneficial to have node.js running the backend for our system which relies on constantly getting and sending information to many thousands of users in a concurrent enviroment. This server may be used in conjunction with other servers in which it handles the Events Module. Another technology that could be used is MongoDB for the database storage in the NavUp system aswell as the database storage and retrieval of information. Along with MongoDb being a nosql database which stream storage and retrieval can benefit from, it also utilizes concurrency with reads and is therefore the efficient approach for data storage for our NavUpSystem.
\end{document}

\subsection{Navigation Module}
Leaflet is an open-source JavaScript library that provides mapping functions for mobile developers. It runs efficiently on all major mobile and desktop platforms and therefore satisfies our requirement of the NavUP system needing to have Android, iOS and a web-based frontend. Leaflet can work with custom GIS data so will allow for the use of our provided GIS mapping of the University of Pretoria’s campus. The API at a basic level is mainly for displaying map data however there are a number of plugins available to expand the functionality. One of these is called Leaflet Routing Machine and provides a graphical extension to leaflet, displaying routes between a start and end point. Leaflet routing machine provides additional functionality for backend routing engines, one of these being OSRM (Open Source Routing Machine). OSRM uses extremely fast methods for shortest path routing and provides turn by turn navigation instructions to the user. These three technologies already interface with one another, are all open source and satisfy most of the constraints of the NavUP Navigation module. Additional implementation could ensure that the technologies integrate well into the NavUP system.
\end
